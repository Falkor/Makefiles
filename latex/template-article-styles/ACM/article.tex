% =============================================================================
% .tex --
% Author(s): Sebastien Varrette      <Sebastien.Varrette@uni.lu>
%
% More information on LaTeX: http://www.latex-project.org/
% LaTeX symbol list:
% http://www.ctan.org/tex-archive/info/symbols/comprehensive/symbols-a4.pdf
% =============================================================================
% Based on "sig-alternate.tex" V2.0 May 2012
% This file should be compiled with V2.5 of "sig-alternate.cls" May 2012
%
% This .tex file (and associated .cls V2.5) produces:
% 1) The Permission Statement
% 2) The Conference (location) Info information
% 3) The Copyright Line with ACM data
% 4) NO page numbers
%
% Using 'sig-alternate.cls' you have control, however, from within
% the source .tex file, over both the CopyrightYear
% (defaulted to 200X) and the ACM Copyright Data
% (defaulted to X-XXXXX-XX-X/XX/XX).
% e.g.
% \CopyrightYear{2007} will cause 2007 to appear in the copyright line.
% \crdata{0-12345-67-8/90/12} will cause 0-12345-67-8/90/12 to appear in the copyright line.
%
% REMEMBER HOWEVER: After having produced the .bbl file,
% and prior to final submission, you *NEED* to 'insert'
% your .bbl file into your source .tex file so as to provide
% ONE 'self-contained' source file.
%
% ================= IF YOU HAVE QUESTIONS =======================
% Questions regarding the SIGS styles, SIGS policies and
% procedures, Conferences etc. should be sent to
% Adrienne Griscti (griscti@acm.org)
%
% Technical questions _only_ to
% Gerald Murray (murray@hq.acm.org)
% ===============================================================
%
% For tracking purposes - this is V2.0 - May 2012

\documentclass{sig-alternate}
\usepackage{_style}

% cf http://www.acm.org/sigs/publications/sigfaq#a17
\def\sharedaffiliation{%
\end{tabular}
\begin{tabular}{c}}

\begin{document}

% --- Author Metadata here ---
\conferenceinfo{SC}{'13 Denver, USA}
\CopyrightYear{2013} % Allows default copyright year (20XX) to be over-ridden - IF NEED BE.
% \crdata{0-12345-67-8/90/01}  % Allows default copyright data (0-89791-88-6/97/05) to be over-ridden - IF NEED BE
% --- End of Author Metadata ---

\title{My title\\ subline title}
% \subtitle{[Extended Abstract]
% \titlenote{A full version of this paper is available as
% \textit{Author's Guide to Preparing ACM SIG Proceedings Using
% \LaTeX$2_\epsilon$\ and BibTeX} at
% \texttt{www.acm.org/eaddress.htm}}}

\numberofauthors{5}
\author{
% 1st. author
  \alignauthor
  S\'ebastien Varrette$^{1,2}$\\
  \email{\url{Sebastien.Varrette@uni.lu}}
  % 2nd. author
  \alignauthor
  Xavier Besseron$^{1}$\\
  \email{\url{Xavier.Besseron@uni.lu}}
  %
  \sharedaffiliation
  \affaddr{$^{1}$Computer Science and Communications (CSC) Research Unit}\\
  \affaddr{$^{2}$Interdisciplinary Centre for Security Reliability and Trust}\\
  \affaddr{6, rue Richard Coudenhove-Kalergi}\\
  \affaddr{L-1359 Luxembourg, Luxembourg}\\
  % \email{Pascal.Bouvry@uni.lu}
}
%\date{24th May 2007}
\maketitle

\begin{abstract}
    %=============================================================================
% _abstract.tex --             
%=============================================================================


abstract goes here





%~~~~~~~~~~~~~~~~~~~~~~~~~~~~~~~~~~~~~~~~~~~~~~~~~~~~~~~~~~~~~~~~
% eof
%
% Local Variables:
% mode: latex
% mode: flyspell
% mode: auto-fill
% fill-column: 80
% End:

\end{abstract}

% A category with the (minimum) three required fields
%\category{H.4}{Information Systems Applications}{Miscellaneous}
%\category{D.4.1}{Performance, Analysis & Tools (PAT)}
% %A category including the fourth, optional field follows...
\category{D.2.8}{Software Engineering}{Metrics}[complexity measures, performance measures]

\terms{HPC, Performance Evaluation}

\keywords{
    % Performance evaluation,
    Keyword1,
    Keyword2,
    Keyword3
}

% ===================================
\section{Introduction}
\label{sec:introduction}
.template/_introduction.tex

% ========================================
\section{Context \& Motivations}
\label{sec:context}
.template/_context.tex

% ========================================
\section{Implementation and Experimental Setup}
\label{sec:implem}\label{sec:experimental_setup}
.template/_implem.tex

% ====================================================================
\section{Validation and Experimental Results}
\label{sec:experiments}
.template/_experiments.tex

% ====================================================================
\section{Related Work}
\label{sec:related_works}
.template/_related_works.tex

% ====================================================================
\section{Conclusion}
\label{sec:conclusion}
.template/_conclusion.tex

~\\
{\noindent \textbf{Acknowledgments:}}
% \\
The experiments presented in this paper were carried
out using the HPC facility of the University of Luxembourg.

\bibliographystyle{abbrv}
\bibliography{biblio}

\appendix

% \section{Acronyms used}
.template/_acronyms.tex

% that's all folks
\end{document}


%~~~~~~~~~~~~~~~~~~~~~~~~~~~~~~~~~~~~~~~~~~~~~~~~~~~~~~~~~~~~~~~~
% eof
%
% Local Variables:
% mode: latex
% mode: flyspell
% mode: auto-fill
% fill-column: 80
% End:
